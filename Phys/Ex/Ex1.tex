\documentclass{leptc}
\begin{document}







\chap{形式逻辑}

\ent[formal logic]{形式逻辑}

\ent[organon]{工具论}
逻辑学既非理论知识,也非实际知识,而是知识的工具

\ent[concept]{概念}
\ent[property / attribute]{性质|属性}
\ent{矛盾} 自洽

\ent[substantial definition]{实质定义} \hspace{40pt}
\eng[genus plus species difference]{用属加种差的方法来定义}

\ent[nominal definition]{语词定义}

\ent[proposition]{命题} 具有真假意义的陈述语句

\ent[deductive inference]{演绎推理} 由一般到特殊
\com{如 \ent[syllogism]{三段论}}
\ent[inductive inference]{归纳推理} 由特殊到一般
\quad
推理的全过程叫作 \ent[proof]{证明}

\ent[axiom / postulation]{公理|公设}
不证自明,无法推导,推导的起点

\ent[axiomatic system]{公理系统}
由若干公理经演绎推理形成的自洽的命题体系\com{公理化}

\ent[theorem / corollary]{定理|推论}
由公理及其它已知为真的命题经逻辑推理证明为真的重要命题
\ent[lemma]{引理}中间命题

\ent[law / principle / rule]{定律|原理|规则}
观察总结出来的客观规律,由长期实践的事实所证明,
在人类认知范围内普遍适用

\ent[conjecture / hypothesis]{猜想|假说}
可能为真但未被证明的命题
\com{当它被证明为真后便是定理}

\ent[paradox]{悖论} 命题成立则推出其否定也成立 \eq{p\ns\neg p}

\ent[sophistry]{诡辩}
循环论证,机械类比,以偏概全,偷换论题,偷换概念,模棱两可

\ent[fallacy]{谬误} 形式谬误:推理不正确,
非形式谬误:语言歧义,不合事实

\ent[counter example]{反例} 命题 \eq{A\rightarrow B} 中
满足条件 \eq{A} 但不满足结论 \eq{B} 的实例


% \enl{例} 假推出真:若条件是假命题 \com{假如一个永远不会发生的事},无论任何结论,命题都为真



\ent{基本假设} 只能由其推出的实验现象来验证


\ \\

\ent{物理学}是探讨物质的结构和运动基本规律的学科
\com{更着重于物质世界普遍而基本的规律}

\ent[science]{科学}认识世界,解决理论问题
\ent[technology]{技术}改造世界,解决实践问题

\ent[modern physics]{现代物理学}
是一门理论和实验高度结合的精确科学

\N1提出问题\com{新现象或新推导}
\N2推测答案\com{建立唯象模型,定性或定量解释}
\N3理论预言\com{可证伪}
\N4实验检验
\N5修改理论\com{确定成立范围}

新的理论必须提出能够为实验所 \ent[falsify]{证伪} 的预言
\com{不说证实是因为找不到反例不是有效证明}

理论不唯一,一个理论包含的假设越少,越简洁,与之符合的事实越多,越普遍,理论就越好


\ \\

费曼做理论5步:
\N1靠直觉猜\N2验证已知例子\N3应用到未知问题,与实验比较\N4有理有据地证明

\com{I know much more than I can prove.}




\end{document}
